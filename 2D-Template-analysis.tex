% Options for packages loaded elsewhere
\PassOptionsToPackage{unicode}{hyperref}
\PassOptionsToPackage{hyphens}{url}
%
\documentclass[
]{article}
\usepackage{amsmath,amssymb}
\usepackage{lmodern}
\usepackage{iftex}
\ifPDFTeX
  \usepackage[T1]{fontenc}
  \usepackage[utf8]{inputenc}
  \usepackage{textcomp} % provide euro and other symbols
\else % if luatex or xetex
  \usepackage{unicode-math}
  \defaultfontfeatures{Scale=MatchLowercase}
  \defaultfontfeatures[\rmfamily]{Ligatures=TeX,Scale=1}
\fi
% Use upquote if available, for straight quotes in verbatim environments
\IfFileExists{upquote.sty}{\usepackage{upquote}}{}
\IfFileExists{microtype.sty}{% use microtype if available
  \usepackage[]{microtype}
  \UseMicrotypeSet[protrusion]{basicmath} % disable protrusion for tt fonts
}{}
\makeatletter
\@ifundefined{KOMAClassName}{% if non-KOMA class
  \IfFileExists{parskip.sty}{%
    \usepackage{parskip}
  }{% else
    \setlength{\parindent}{0pt}
    \setlength{\parskip}{6pt plus 2pt minus 1pt}}
}{% if KOMA class
  \KOMAoptions{parskip=half}}
\makeatother
\usepackage{xcolor}
\IfFileExists{xurl.sty}{\usepackage{xurl}}{} % add URL line breaks if available
\IfFileExists{bookmark.sty}{\usepackage{bookmark}}{\usepackage{hyperref}}
\hypersetup{
  pdftitle={Template for 2D data analysis with INLABru},
  pdfauthor={J Matthiopoulos (collated from INLABru vignettes)},
  hidelinks,
  pdfcreator={LaTeX via pandoc}}
\urlstyle{same} % disable monospaced font for URLs
\usepackage[margin=1in]{geometry}
\usepackage{color}
\usepackage{fancyvrb}
\newcommand{\VerbBar}{|}
\newcommand{\VERB}{\Verb[commandchars=\\\{\}]}
\DefineVerbatimEnvironment{Highlighting}{Verbatim}{commandchars=\\\{\}}
% Add ',fontsize=\small' for more characters per line
\usepackage{framed}
\definecolor{shadecolor}{RGB}{248,248,248}
\newenvironment{Shaded}{\begin{snugshade}}{\end{snugshade}}
\newcommand{\AlertTok}[1]{\textcolor[rgb]{0.94,0.16,0.16}{#1}}
\newcommand{\AnnotationTok}[1]{\textcolor[rgb]{0.56,0.35,0.01}{\textbf{\textit{#1}}}}
\newcommand{\AttributeTok}[1]{\textcolor[rgb]{0.77,0.63,0.00}{#1}}
\newcommand{\BaseNTok}[1]{\textcolor[rgb]{0.00,0.00,0.81}{#1}}
\newcommand{\BuiltInTok}[1]{#1}
\newcommand{\CharTok}[1]{\textcolor[rgb]{0.31,0.60,0.02}{#1}}
\newcommand{\CommentTok}[1]{\textcolor[rgb]{0.56,0.35,0.01}{\textit{#1}}}
\newcommand{\CommentVarTok}[1]{\textcolor[rgb]{0.56,0.35,0.01}{\textbf{\textit{#1}}}}
\newcommand{\ConstantTok}[1]{\textcolor[rgb]{0.00,0.00,0.00}{#1}}
\newcommand{\ControlFlowTok}[1]{\textcolor[rgb]{0.13,0.29,0.53}{\textbf{#1}}}
\newcommand{\DataTypeTok}[1]{\textcolor[rgb]{0.13,0.29,0.53}{#1}}
\newcommand{\DecValTok}[1]{\textcolor[rgb]{0.00,0.00,0.81}{#1}}
\newcommand{\DocumentationTok}[1]{\textcolor[rgb]{0.56,0.35,0.01}{\textbf{\textit{#1}}}}
\newcommand{\ErrorTok}[1]{\textcolor[rgb]{0.64,0.00,0.00}{\textbf{#1}}}
\newcommand{\ExtensionTok}[1]{#1}
\newcommand{\FloatTok}[1]{\textcolor[rgb]{0.00,0.00,0.81}{#1}}
\newcommand{\FunctionTok}[1]{\textcolor[rgb]{0.00,0.00,0.00}{#1}}
\newcommand{\ImportTok}[1]{#1}
\newcommand{\InformationTok}[1]{\textcolor[rgb]{0.56,0.35,0.01}{\textbf{\textit{#1}}}}
\newcommand{\KeywordTok}[1]{\textcolor[rgb]{0.13,0.29,0.53}{\textbf{#1}}}
\newcommand{\NormalTok}[1]{#1}
\newcommand{\OperatorTok}[1]{\textcolor[rgb]{0.81,0.36,0.00}{\textbf{#1}}}
\newcommand{\OtherTok}[1]{\textcolor[rgb]{0.56,0.35,0.01}{#1}}
\newcommand{\PreprocessorTok}[1]{\textcolor[rgb]{0.56,0.35,0.01}{\textit{#1}}}
\newcommand{\RegionMarkerTok}[1]{#1}
\newcommand{\SpecialCharTok}[1]{\textcolor[rgb]{0.00,0.00,0.00}{#1}}
\newcommand{\SpecialStringTok}[1]{\textcolor[rgb]{0.31,0.60,0.02}{#1}}
\newcommand{\StringTok}[1]{\textcolor[rgb]{0.31,0.60,0.02}{#1}}
\newcommand{\VariableTok}[1]{\textcolor[rgb]{0.00,0.00,0.00}{#1}}
\newcommand{\VerbatimStringTok}[1]{\textcolor[rgb]{0.31,0.60,0.02}{#1}}
\newcommand{\WarningTok}[1]{\textcolor[rgb]{0.56,0.35,0.01}{\textbf{\textit{#1}}}}
\usepackage{longtable,booktabs,array}
\usepackage{calc} % for calculating minipage widths
% Correct order of tables after \paragraph or \subparagraph
\usepackage{etoolbox}
\makeatletter
\patchcmd\longtable{\par}{\if@noskipsec\mbox{}\fi\par}{}{}
\makeatother
% Allow footnotes in longtable head/foot
\IfFileExists{footnotehyper.sty}{\usepackage{footnotehyper}}{\usepackage{footnote}}
\makesavenoteenv{longtable}
\usepackage{graphicx}
\makeatletter
\def\maxwidth{\ifdim\Gin@nat@width>\linewidth\linewidth\else\Gin@nat@width\fi}
\def\maxheight{\ifdim\Gin@nat@height>\textheight\textheight\else\Gin@nat@height\fi}
\makeatother
% Scale images if necessary, so that they will not overflow the page
% margins by default, and it is still possible to overwrite the defaults
% using explicit options in \includegraphics[width, height, ...]{}
\setkeys{Gin}{width=\maxwidth,height=\maxheight,keepaspectratio}
% Set default figure placement to htbp
\makeatletter
\def\fps@figure{htbp}
\makeatother
\setlength{\emergencystretch}{3em} % prevent overfull lines
\providecommand{\tightlist}{%
  \setlength{\itemsep}{0pt}\setlength{\parskip}{0pt}}
\setcounter{secnumdepth}{-\maxdimen} % remove section numbering
\ifLuaTeX
  \usepackage{selnolig}  % disable illegal ligatures
\fi

\title{Template for 2D data analysis with INLABru}
\author{J Matthiopoulos (collated from INLABru vignettes)}
\date{2022-05-27}

\begin{document}
\maketitle

\hypertarget{i.-preparation}{%
\subsection{I. Preparation}\label{i.-preparation}}

\hypertarget{i.1.-load-libraries}{%
\subsubsection{I.1. Load libraries}\label{i.1.-load-libraries}}

\begin{Shaded}
\begin{Highlighting}[]
\CommentTok{\# Essential}
\FunctionTok{library}\NormalTok{(inlabru)}
\FunctionTok{library}\NormalTok{(INLA)}
\FunctionTok{bru\_options\_set}\NormalTok{(}\AttributeTok{inla.mode =} \StringTok{"experimental"}\NormalTok{)}

\CommentTok{\# Visualisation}
\FunctionTok{library}\NormalTok{(ggplot2)}
\FunctionTok{library}\NormalTok{(RColorBrewer)}

\CommentTok{\# Loaded dependencies}
\CommentTok{\#library(sp)}
\CommentTok{\#library(Matrix)}
\CommentTok{\#library(foreach)}
\CommentTok{\#library(parallel)}

\CommentTok{\# Optional}
\FunctionTok{library}\NormalTok{(mgcv) }\CommentTok{\# For independent model performance comparisons, used as an exact method}


\CommentTok{\#library(sf)}
\CommentTok{\#library(raster)}
\CommentTok{\#library(rmapshaper)}
\CommentTok{\#library(tidyr)}
\end{Highlighting}
\end{Shaded}

\hypertarget{i.2.-load-data}{%
\subsubsection{I.2. Load data}\label{i.2.-load-data}}

In the example below, it is assumed that the data reside in a package,
such as `inlabru'. The `try' option explores the list of available
datasets. The second line loads the particular one. Other ways of
importing the data, assuming they are not in a package (`?data') with
option `lib.loc' for pathname.

\begin{Shaded}
\begin{Highlighting}[]
\CommentTok{\#try(data(package="inlabru"))}
\FunctionTok{data}\NormalTok{(gorillas, }\AttributeTok{package =} \StringTok{"inlabru"}\NormalTok{)}
\end{Highlighting}
\end{Shaded}

\hypertarget{i.3.-ensure-data-formatting}{%
\subsubsection{I.3. Ensure data
formatting}\label{i.3.-ensure-data-formatting}}

The overall structure of the data can be explored by `str()'. The point
locations (here, `nests') need to be a `SpatialPointsDataFrame'. If the
point data are not in this form then, they will need to be converted by
providing an appropriate spatial projection (`?SpatialPointsDataFrame').

\begin{Shaded}
\begin{Highlighting}[]
\CommentTok{\#str(gorillas)}
\CommentTok{\#str(gorillas$nests)}
\NormalTok{myPoints}\OtherTok{\textless{}{-}}\NormalTok{gorillas}\SpecialCharTok{$}\NormalTok{nests }\CommentTok{\# assign to shorthand}
\NormalTok{myCovs }\OtherTok{\textless{}{-}}\NormalTok{ gorillas}\SpecialCharTok{$}\NormalTok{gcov }\CommentTok{\# Covariate data}
\end{Highlighting}
\end{Shaded}

Here, the data set comes with in-built mesh and boundary components, but
details on mesh specification are in Section III.1.

\begin{Shaded}
\begin{Highlighting}[]
\CommentTok{\#str(gorillas$mesh)}
\CommentTok{\#str(gorillas$boundary)}
\NormalTok{myMesh}\OtherTok{\textless{}{-}}\NormalTok{gorillas}\SpecialCharTok{$}\NormalTok{mesh }\CommentTok{\# assign to shorthand}
\NormalTok{myBoundary}\OtherTok{\textless{}{-}}\NormalTok{gorillas}\SpecialCharTok{$}\NormalTok{boundary }\CommentTok{\# assign to shorthand}
\end{Highlighting}
\end{Shaded}

Visualise the factor covariate data (\texttt{vegetation} type)

\begin{Shaded}
\begin{Highlighting}[]
\FunctionTok{ggplot}\NormalTok{() }\SpecialCharTok{+}
  \FunctionTok{gg}\NormalTok{(myCovs}\SpecialCharTok{$}\NormalTok{vegetation) }\SpecialCharTok{+}
  \FunctionTok{gg}\NormalTok{(myMesh) }\SpecialCharTok{+}
  \FunctionTok{gg}\NormalTok{(myBoundary) }\SpecialCharTok{+}
  \FunctionTok{gg}\NormalTok{(myPoints, }\AttributeTok{color =} \StringTok{"white"}\NormalTok{, }\AttributeTok{cex =} \FloatTok{0.5}\NormalTok{) }\SpecialCharTok{+}
  \FunctionTok{coord\_equal}\NormalTok{()}
\end{Highlighting}
\end{Shaded}

\includegraphics{2D-Template-analysis_files/figure-latex/unnamed-chunk-5-1.pdf}

\hypertarget{ii.-glms}{%
\subsection{II. GLMs}\label{ii.-glms}}

\hypertarget{ii.1-a-glm-with-a-factor-covariate-only}{%
\subsubsection{II.1 A GLM with a factor covariate
only}\label{ii.1-a-glm-with-a-factor-covariate-only}}

-----\textgreater{} Preparation of a factor covariate layer on a mesh is
likely to be challenging. Most data layers will come in raster form and
while the projection to a mesh is easy for continuous variables (e.g.,
\texttt{pixel\_n\ \textless{}-\ raster(spatialpixeldf)}), this is not
straightforward with factors. Of course, in the data above this comes
pre-made. As a result, attempting to run a model using this covariate
data together with a custom mesh created below is not going to work.

To construct a model with vegetation type as a fixed effect, we need to
tell `lgcp' how to find the vegetation type at any point in space, and
we do this by creating model components with a fixed effect that we call
\texttt{vegetation} (we could call it anything), as follows:

\begin{Shaded}
\begin{Highlighting}[]
\NormalTok{myFactComp }\OtherTok{\textless{}{-}}\NormalTok{ coordinates }\SpecialCharTok{\textasciitilde{}} \FunctionTok{vegetation}\NormalTok{(myCovs}\SpecialCharTok{$}\NormalTok{vegetation, }\AttributeTok{model =} \StringTok{"factor\_full"}\NormalTok{) }\SpecialCharTok{{-}} \DecValTok{1}
\end{Highlighting}
\end{Shaded}

Notes:

\begin{itemize}
\tightlist
\item
  We need to tell `lgcp' that this is a factor fixed effect, which we do
  with \texttt{model="factor\_full"}, giving one coefficient for each
  factor level.
\item
  We need to be careful about overparameterisation when using factors.
  Unlike regression models like \texttt{lm()}, \texttt{glm()} or
  \texttt{gam()}, \texttt{lgcp()}, \texttt{inlabru} does not
  automatically remove the first level and absorb it into an intercept.
  Instead, we can either use \texttt{model="factor\_full"} without an
  intercept, or \texttt{model="factor\_contrast"}, which does remove the
  first level.
\end{itemize}

\begin{Shaded}
\begin{Highlighting}[]
\NormalTok{myFactCompAlt }\OtherTok{\textless{}{-}}\NormalTok{ coordinates }\SpecialCharTok{\textasciitilde{}} \FunctionTok{vegetation}\NormalTok{(myCovs}\SpecialCharTok{$}\NormalTok{vegetation, }\AttributeTok{model =} \StringTok{"factor\_contrast"}\NormalTok{) }\SpecialCharTok{+} \FunctionTok{Intercept}\NormalTok{(}\DecValTok{1}\NormalTok{)}
\end{Highlighting}
\end{Shaded}

The model can be fitted as follows:

\begin{Shaded}
\begin{Highlighting}[]
\NormalTok{myFactorGLM }\OtherTok{\textless{}{-}} \FunctionTok{lgcp}\NormalTok{(myFactComp, myPoints, }\AttributeTok{samplers =}\NormalTok{ myBoundary, }\AttributeTok{domain =} \FunctionTok{list}\NormalTok{(}\AttributeTok{coordinates =}\NormalTok{ myMesh))}
\end{Highlighting}
\end{Shaded}

To predict the intensity, and plot the median intensity surface, the
\texttt{predidct} function of \texttt{inlabru} takes into its
\texttt{data} argument a \texttt{SpatialPointsDataFrame}, a
\texttt{SpatialPixelsDataFrame} or a \texttt{data.frame}. We can use the
\texttt{inlabru} function \texttt{pixels} to generate a
\texttt{SpatialPixelsDataFrame} only within the boundary, using its
\texttt{mask} argument, as shown below.

\begin{Shaded}
\begin{Highlighting}[]
\NormalTok{df }\OtherTok{\textless{}{-}} \FunctionTok{pixels}\NormalTok{(myMesh, }\AttributeTok{mask =}\NormalTok{ myBoundary)}
\NormalTok{int1 }\OtherTok{\textless{}{-}} \FunctionTok{predict}\NormalTok{(myFactorGLM, }\AttributeTok{data =}\NormalTok{ df, }\SpecialCharTok{\textasciitilde{}} \FunctionTok{exp}\NormalTok{(vegetation))}
\FunctionTok{ggplot}\NormalTok{() }\SpecialCharTok{+}
  \FunctionTok{gg}\NormalTok{(int1) }\SpecialCharTok{+}
  \FunctionTok{gg}\NormalTok{(myBoundary, }\AttributeTok{alpha =} \DecValTok{0}\NormalTok{, }\AttributeTok{lwd =} \DecValTok{2}\NormalTok{) }\SpecialCharTok{+}
  \FunctionTok{gg}\NormalTok{(myPoints, }\AttributeTok{color =} \StringTok{"DarkGreen"}\NormalTok{) }\SpecialCharTok{+}
  \FunctionTok{coord\_equal}\NormalTok{()}
\end{Highlighting}
\end{Shaded}

\includegraphics{2D-Template-analysis_files/figure-latex/unnamed-chunk-9-1.pdf}

The estimated total abundance of points (but not full posterior) can be
obtained as follows.The integration \texttt{weight} values (the
quadrature points) are contained in the \texttt{ipoints} output.

\begin{Shaded}
\begin{Highlighting}[]
\NormalTok{ips }\OtherTok{\textless{}{-}} \FunctionTok{ipoints}\NormalTok{(myBoundary, myMesh)}
\NormalTok{Lambda1 }\OtherTok{\textless{}{-}} \FunctionTok{predict}\NormalTok{(myFactorGLM, ips, }\SpecialCharTok{\textasciitilde{}} \FunctionTok{sum}\NormalTok{(weight }\SpecialCharTok{*} \FunctionTok{exp}\NormalTok{(vegetation)))}
\NormalTok{Lambda1}
\end{Highlighting}
\end{Shaded}

\begin{verbatim}
##       mean       sd   q0.025   median   q0.975     smin     smax         cv
## 1 644.1787 24.60183 594.7933 643.8525 697.7191 579.3538 707.1418 0.03819101
##        var
## 1 605.2502
\end{verbatim}

\hypertarget{ii.1-a-glm-with-a-continuous-covariate-only}{%
\subsubsection{II.1 A GLM with a continuous covariate
only}\label{ii.1-a-glm-with-a-continuous-covariate-only}}

Now lets try a model with elevation as a (continuous) explanatory
variable. (First centre elevations for more stable fitting.)

\begin{Shaded}
\begin{Highlighting}[]
\NormalTok{elev }\OtherTok{\textless{}{-}}\NormalTok{ myCovs}\SpecialCharTok{$}\NormalTok{elevation}
\NormalTok{elev}\SpecialCharTok{$}\NormalTok{elevation }\OtherTok{\textless{}{-}}\NormalTok{ elev}\SpecialCharTok{$}\NormalTok{elevation }\SpecialCharTok{{-}} \FunctionTok{mean}\NormalTok{(elev}\SpecialCharTok{$}\NormalTok{elevation, }\AttributeTok{na.rm =} \ConstantTok{TRUE}\NormalTok{)}

\FunctionTok{ggplot}\NormalTok{() }\SpecialCharTok{+}
  \FunctionTok{gg}\NormalTok{(elev) }\SpecialCharTok{+}
  \FunctionTok{gg}\NormalTok{(myBoundary, }\AttributeTok{alpha =} \DecValTok{0}\NormalTok{) }\SpecialCharTok{+}
  \FunctionTok{coord\_fixed}\NormalTok{()}
\end{Highlighting}
\end{Shaded}

\includegraphics{2D-Template-analysis_files/figure-latex/unnamed-chunk-11-1.pdf}

The elevation variable here is of class \texttt{SpatialGridDataFrame},
that can be handled in the same way as the vegetation covariate.
However, since in some cases data may be stored differently, and other
methods are needed to access the stored values. In such cases, we can
define a function that knows how to evaluate the covariate at arbitrary
points in the survey region, and call that function in the component
definition. In this case, we can use a powerful method from the `sp'
package to do this. We use this to create the needed function.

\begin{Shaded}
\begin{Highlighting}[]
\NormalTok{f.elev }\OtherTok{\textless{}{-}} \ControlFlowTok{function}\NormalTok{(x, y) \{}
  \CommentTok{\# turn coordinates into SpatialPoints object:}
  \CommentTok{\# with the appropriate coordinate reference system (CRS)}
\NormalTok{  spp }\OtherTok{\textless{}{-}} \FunctionTok{SpatialPoints}\NormalTok{(}\FunctionTok{data.frame}\NormalTok{(}\AttributeTok{x =}\NormalTok{ x, }\AttributeTok{y =}\NormalTok{ y), }\AttributeTok{proj4string =} \FunctionTok{fm\_sp\_get\_crs}\NormalTok{(elev))}
  \FunctionTok{proj4string}\NormalTok{(spp) }\OtherTok{\textless{}{-}} \FunctionTok{fm\_sp\_get\_crs}\NormalTok{(elev)}
  \CommentTok{\# Extract elevation values at spp coords, from our elev SpatialGridDataFrame}
\NormalTok{  v }\OtherTok{\textless{}{-}} \FunctionTok{over}\NormalTok{(spp, elev)}
  \ControlFlowTok{if}\NormalTok{ (}\FunctionTok{any}\NormalTok{(}\FunctionTok{is.na}\NormalTok{(v}\SpecialCharTok{$}\NormalTok{elevation))) \{}
\NormalTok{    v}\SpecialCharTok{$}\NormalTok{elevation }\OtherTok{\textless{}{-}}\NormalTok{ inlabru}\SpecialCharTok{:::}\FunctionTok{bru\_fill\_missing}\NormalTok{(elev, spp, v}\SpecialCharTok{$}\NormalTok{elevation)}
\NormalTok{  \}}
  \FunctionTok{return}\NormalTok{(v}\SpecialCharTok{$}\NormalTok{elevation)}
\NormalTok{\}}
\end{Highlighting}
\end{Shaded}

The model is fitted as follows:

\begin{Shaded}
\begin{Highlighting}[]
\NormalTok{ecomp }\OtherTok{\textless{}{-}}\NormalTok{ coordinates }\SpecialCharTok{\textasciitilde{}} \FunctionTok{elev}\NormalTok{(}\FunctionTok{f.elev}\NormalTok{(x, y), }\AttributeTok{model =} \StringTok{"linear"}\NormalTok{) }\SpecialCharTok{+} \FunctionTok{Intercept}\NormalTok{(}\DecValTok{1}\NormalTok{)}
\NormalTok{myVarGLM}\OtherTok{\textless{}{-}} \FunctionTok{lgcp}\NormalTok{(ecomp, myPoints, }\AttributeTok{samplers =}\NormalTok{ myBoundary, }\AttributeTok{domain =} \FunctionTok{list}\NormalTok{(}\AttributeTok{coordinates =}\NormalTok{ myMesh))}
\FunctionTok{summary}\NormalTok{(myVarGLM)}
\end{Highlighting}
\end{Shaded}

\begin{verbatim}
## inlabru version: 2.5.2
## INLA version: 22.05.07
## Components:
##   elev: Model types main='linear', group='exchangeable', replicate='iid'
##   Intercept: Model types main='linear', group='exchangeable', replicate='iid'
## Likelihoods:
##   Family: 'cp'
##     Data class: 'SpatialPointsDataFrame'
##     Predictor: coordinates ~ .
## Time used:
##     Pre = 0.331, Running = 0.13, Post = 0.0343, Total = 0.495 
## Fixed effects:
##            mean    sd 0.025quant 0.5quant 0.975quant mode kld
## elev      0.004 0.000      0.004    0.004      0.005   NA   0
## Intercept 3.068 0.056      2.959    3.068      3.177   NA   0
## 
## Deviance Information Criterion (DIC) ...............: -952.12
## Deviance Information Criterion (DIC, saturated) ....: -17729.16
## Effective number of parameters .....................: -2246.11
## 
## Watanabe-Akaike information criterion (WAIC) ...: 1368.06
## Effective number of parameters .................: 1.94
## 
## Marginal log-Likelihood:  -1788.51 
##  is computed 
## Posterior summaries for the linear predictor and the fitted values are computed
## (Posterior marginals needs also 'control.compute=list(return.marginals.predictor=TRUE)')
\end{verbatim}

Predictions are made in same way as with factor variable

\begin{Shaded}
\begin{Highlighting}[]
\NormalTok{df }\OtherTok{\textless{}{-}} \FunctionTok{pixels}\NormalTok{(myMesh, }\AttributeTok{mask =}\NormalTok{ myBoundary)}
\NormalTok{int1 }\OtherTok{\textless{}{-}} \FunctionTok{predict}\NormalTok{(myVarGLM, }\AttributeTok{data =}\NormalTok{ df, }\SpecialCharTok{\textasciitilde{}} \FunctionTok{exp}\NormalTok{(elev))}
\FunctionTok{ggplot}\NormalTok{() }\SpecialCharTok{+}
  \FunctionTok{gg}\NormalTok{(int1) }\SpecialCharTok{+}
  \FunctionTok{gg}\NormalTok{(myBoundary, }\AttributeTok{alpha =} \DecValTok{0}\NormalTok{, }\AttributeTok{lwd =} \DecValTok{2}\NormalTok{) }\SpecialCharTok{+}
  \FunctionTok{gg}\NormalTok{(myPoints, }\AttributeTok{color =} \StringTok{"DarkGreen"}\NormalTok{) }\SpecialCharTok{+}
  \FunctionTok{coord\_equal}\NormalTok{()}
\end{Highlighting}
\end{Shaded}

\includegraphics{2D-Template-analysis_files/figure-latex/unnamed-chunk-14-1.pdf}

\hypertarget{iii.-spde-model-no-covariates}{%
\subsection{III. SPDE model, no
covariates}\label{iii.-spde-model-no-covariates}}

\hypertarget{iii.1.-build-the-spde-mesh}{%
\subsubsection{III.1. Build the SPDE
mesh}\label{iii.1.-build-the-spde-mesh}}

For this section I will ignore the fact that the gorillas data set comes
with a predefined mesh and develop one from scratch. There are several
arguments that can be used to build the mesh. The arguments for a
two-dimensional mesh construction are the following:

\begin{Shaded}
\begin{Highlighting}[]
\FunctionTok{args}\NormalTok{(inla.mesh}\FloatTok{.2}\NormalTok{d)}
\end{Highlighting}
\end{Shaded}

\begin{verbatim}
## function (loc = NULL, loc.domain = NULL, offset = NULL, n = NULL, 
##     boundary = NULL, interior = NULL, max.edge = NULL, min.angle = NULL, 
##     cutoff = 1e-12, max.n.strict = NULL, max.n = NULL, plot.delay = NULL, 
##     crs = NULL) 
## NULL
\end{verbatim}

First, some reference about the study region is needed, which can be
provided by either:

\begin{enumerate}
\def\labelenumi{\arabic{enumi}.}
\tightlist
\item
  The location of points, supplied on the \texttt{loc} argument
  \footnote{Matrix of point locations to be used as initial
    triangulation nodes. Can alternatively be a \texttt{SpatialPoints}
    or \texttt{SpatialPointsDataFrame} object.}.
\item
  A boundary of the region defined by a set of polygons (e.g., a polygon
  defining the coastline of the study) supplied on the \texttt{boundary}
  argument.
\item
  The domain extent which can be supplied as a single polygon on the
  \texttt{loc.domain} argument.
\end{enumerate}

Note that if either (1) the location of points or (3) the domain extent
are specified, the mesh will be constructed based on a convex hull (a
polygon of triangles out of the domain area). Alternatively, it is
possible to include a non-convex hull as a boundary in the mesh
construction instead of the \texttt{loc} or \texttt{loc.domain}
arguments. This will result in the triangulation to be constrained by
the boundary. A non-convex hull mesh can also be created by building a
boundary for the points using the \texttt{inla.nonconvex.hull()}
function. Finally, the other compulsory argument that needs to be
specified is \texttt{max.edge} which determines the largest allowed
triangle length (the lower the value for max.edge the higher the
resolution). The value supplied to this argument can be either a scalar,
in which case the value controls the triangle edge lengths in the inner
domain, or a length two vector that controls the edge lengths in the
inner domain and in the outer extension respectively. Notice that The
value (or values) passed to the \texttt{max.edge} option must be on the
same scale unit as the coordinates.

While there is no general rule for setting a correct value for the
\texttt{max.edge}, a value for \texttt{max.edge} that is too close to
the spatial range will make the task of fitting a smooth SPDE difficult.
On the other hand, if the \texttt{max.edge} value is too small compared
to the spatial range, the mesh will have a large number of vertices
leading to a more computationally demanding fitting process (which might
not necessarily lead to better results). Thus, it is better to begin the
analysis with a coarse matrix and evaluate the model on a finer grid as
a final step. The \texttt{cutoff} option regulates the minimum length of
each edge (could have been called ``min.edge'', more intuitively?).

I first develop the mesh by using the boundary of the study area. (The
\texttt{ggplot2} function \texttt{coord\_fixed()} sets the aspect ratio,
which defaults to 1.)

\begin{Shaded}
\begin{Highlighting}[]
\CommentTok{\# Build the mesh}
\NormalTok{bbox}\OtherTok{\textless{}{-}}\FunctionTok{bbox}\NormalTok{(myBoundary)}
\NormalTok{max.edge }\OtherTok{\textless{}{-}} \DecValTok{1}\SpecialCharTok{/}\DecValTok{20}\SpecialCharTok{*}\FunctionTok{sqrt}\NormalTok{((bbox[}\DecValTok{1}\NormalTok{,}\DecValTok{1}\NormalTok{]}\SpecialCharTok{{-}}\NormalTok{bbox[}\DecValTok{1}\NormalTok{,}\DecValTok{2}\NormalTok{])}\SpecialCharTok{\^{}}\DecValTok{2}\SpecialCharTok{+}\NormalTok{(bbox[}\DecValTok{2}\NormalTok{,}\DecValTok{1}\NormalTok{]}\SpecialCharTok{{-}}\NormalTok{bbox[}\DecValTok{2}\NormalTok{,}\DecValTok{2}\NormalTok{])}\SpecialCharTok{\^{}}\DecValTok{2}\NormalTok{)}

\NormalTok{myMesh2 }\OtherTok{\textless{}{-}} \FunctionTok{inla.mesh.2d}\NormalTok{(}\AttributeTok{boundary =}\NormalTok{ myBoundary,}
                    \AttributeTok{max.edge =}\NormalTok{ max.edge,}
                    \AttributeTok{cutoff=}\NormalTok{.}\DecValTok{1}\NormalTok{)}

\CommentTok{\#plot(mesh)}

\FunctionTok{ggplot}\NormalTok{() }\SpecialCharTok{+}
     \FunctionTok{gg}\NormalTok{(}\AttributeTok{data=}\NormalTok{myBoundary,}\AttributeTok{color=}\StringTok{\textquotesingle{}turquoise\textquotesingle{}}\NormalTok{,}\AttributeTok{fill=}\StringTok{\textquotesingle{}transparent\textquotesingle{}}\NormalTok{)}\SpecialCharTok{+}  
  \FunctionTok{gg}\NormalTok{(myMesh2)}\SpecialCharTok{+}
  \FunctionTok{gg}\NormalTok{(myPoints) }\SpecialCharTok{+}
  \FunctionTok{coord\_fixed}\NormalTok{() }\SpecialCharTok{+}
  \FunctionTok{ggtitle}\NormalTok{(}\StringTok{"Points"}\NormalTok{)}
\end{Highlighting}
\end{Shaded}

\includegraphics{2D-Template-analysis_files/figure-latex/unnamed-chunk-16-1.pdf}

We can also specify an outer layer with a lower triangle density
(i.e.~where no points occur) to avoid this boundary effect. This can be
done by supplying a vector of two values so that the spatial domain is
divided into an inner and an outer area. Here, we will define the
max.edge such that the outer layer will have a triangle density two
times lower than than the inner layer (i.e.~twice the length for the
outer layer edges).The amount to which the domain should be extended in
the inner and outer part can be controlled with the offset argument of
the inla.mesh.2d function. For this example we will expand the inner
layer by the same amount as the max.edge and the outer layer by the
range we assumed when defining the inner max.edge value (i.e.~1/4 of the
spatial extent).

\begin{Shaded}
\begin{Highlighting}[]
\CommentTok{\# Build the mesh}
\NormalTok{bbox}\OtherTok{\textless{}{-}}\FunctionTok{bbox}\NormalTok{(myBoundary)}
\NormalTok{max.edge }\OtherTok{\textless{}{-}} \DecValTok{1}\SpecialCharTok{/}\DecValTok{20}\SpecialCharTok{*}\FunctionTok{sqrt}\NormalTok{((bbox[}\DecValTok{1}\NormalTok{,}\DecValTok{1}\NormalTok{]}\SpecialCharTok{{-}}\NormalTok{bbox[}\DecValTok{1}\NormalTok{,}\DecValTok{2}\NormalTok{])}\SpecialCharTok{\^{}}\DecValTok{2}\SpecialCharTok{+}\NormalTok{(bbox[}\DecValTok{2}\NormalTok{,}\DecValTok{1}\NormalTok{]}\SpecialCharTok{{-}}\NormalTok{bbox[}\DecValTok{2}\NormalTok{,}\DecValTok{2}\NormalTok{])}\SpecialCharTok{\^{}}\DecValTok{2}\NormalTok{)}
\NormalTok{bound.outer }\OtherTok{\textless{}{-}} \DecValTok{1}\SpecialCharTok{/}\DecValTok{4}\SpecialCharTok{*}\FunctionTok{sqrt}\NormalTok{((bbox[}\DecValTok{1}\NormalTok{,}\DecValTok{1}\NormalTok{]}\SpecialCharTok{{-}}\NormalTok{bbox[}\DecValTok{1}\NormalTok{,}\DecValTok{2}\NormalTok{])}\SpecialCharTok{\^{}}\DecValTok{2}\SpecialCharTok{+}\NormalTok{(bbox[}\DecValTok{2}\NormalTok{,}\DecValTok{1}\NormalTok{]}\SpecialCharTok{{-}}\NormalTok{bbox[}\DecValTok{2}\NormalTok{,}\DecValTok{2}\NormalTok{])}\SpecialCharTok{\^{}}\DecValTok{2}\NormalTok{)}

\NormalTok{myMesh3 }\OtherTok{\textless{}{-}} \FunctionTok{inla.mesh.2d}\NormalTok{(}\AttributeTok{boundary =}\NormalTok{ myBoundary,}
                    \AttributeTok{max.edge =} \FunctionTok{c}\NormalTok{(}\DecValTok{1}\NormalTok{,}\DecValTok{2}\NormalTok{)}\SpecialCharTok{*}\NormalTok{max.edge,}
                    \AttributeTok{cutoff=}\NormalTok{.}\DecValTok{1}\NormalTok{,}
                    \AttributeTok{offset=}\FunctionTok{c}\NormalTok{(max.edge, bound.outer))}

\CommentTok{\#plot(mesh)}

\FunctionTok{ggplot}\NormalTok{() }\SpecialCharTok{+}
     \FunctionTok{gg}\NormalTok{(}\AttributeTok{data=}\NormalTok{myBoundary,}\AttributeTok{color=}\StringTok{\textquotesingle{}turquoise\textquotesingle{}}\NormalTok{,}\AttributeTok{fill=}\StringTok{\textquotesingle{}transparent\textquotesingle{}}\NormalTok{)}\SpecialCharTok{+}  
  \FunctionTok{gg}\NormalTok{(myMesh3)}\SpecialCharTok{+}
  \FunctionTok{gg}\NormalTok{(myPoints) }\SpecialCharTok{+}
  \FunctionTok{coord\_fixed}\NormalTok{() }\SpecialCharTok{+}
  \FunctionTok{ggtitle}\NormalTok{(}\StringTok{"Points"}\NormalTok{)}
\end{Highlighting}
\end{Shaded}

\includegraphics{2D-Template-analysis_files/figure-latex/unnamed-chunk-17-1.pdf}
Alternatively, the mesh can be defined by the observation points.

\begin{Shaded}
\begin{Highlighting}[]
\CommentTok{\# Build the mesh}
\NormalTok{bbox}\OtherTok{\textless{}{-}}\FunctionTok{bbox}\NormalTok{(myBoundary)}
\NormalTok{max.edge }\OtherTok{\textless{}{-}} \DecValTok{1}\SpecialCharTok{/}\DecValTok{20}\SpecialCharTok{*}\FunctionTok{sqrt}\NormalTok{((bbox[}\DecValTok{1}\NormalTok{,}\DecValTok{1}\NormalTok{]}\SpecialCharTok{{-}}\NormalTok{bbox[}\DecValTok{1}\NormalTok{,}\DecValTok{2}\NormalTok{])}\SpecialCharTok{\^{}}\DecValTok{2}\SpecialCharTok{+}\NormalTok{(bbox[}\DecValTok{2}\NormalTok{,}\DecValTok{1}\NormalTok{]}\SpecialCharTok{{-}}\NormalTok{bbox[}\DecValTok{2}\NormalTok{,}\DecValTok{2}\NormalTok{])}\SpecialCharTok{\^{}}\DecValTok{2}\NormalTok{)}
\NormalTok{bound.outer }\OtherTok{\textless{}{-}} \DecValTok{1}\SpecialCharTok{/}\DecValTok{4}\SpecialCharTok{*}\FunctionTok{sqrt}\NormalTok{((bbox[}\DecValTok{1}\NormalTok{,}\DecValTok{1}\NormalTok{]}\SpecialCharTok{{-}}\NormalTok{bbox[}\DecValTok{1}\NormalTok{,}\DecValTok{2}\NormalTok{])}\SpecialCharTok{\^{}}\DecValTok{2}\SpecialCharTok{+}\NormalTok{(bbox[}\DecValTok{2}\NormalTok{,}\DecValTok{1}\NormalTok{]}\SpecialCharTok{{-}}\NormalTok{bbox[}\DecValTok{2}\NormalTok{,}\DecValTok{2}\NormalTok{])}\SpecialCharTok{\^{}}\DecValTok{2}\NormalTok{)}

\NormalTok{myMesh4 }\OtherTok{\textless{}{-}} \FunctionTok{inla.mesh.2d}\NormalTok{(}\AttributeTok{loc =}\NormalTok{ myPoints,}
                    \AttributeTok{max.edge =} \FunctionTok{c}\NormalTok{(}\DecValTok{1}\NormalTok{,}\DecValTok{2}\NormalTok{)}\SpecialCharTok{*}\NormalTok{max.edge,}
                    \AttributeTok{cutoff=}\NormalTok{.}\DecValTok{1}\NormalTok{,}
                    \AttributeTok{offset=}\FunctionTok{c}\NormalTok{(max.edge, bound.outer))}
\CommentTok{\#plot(mesh)}

\FunctionTok{ggplot}\NormalTok{() }\SpecialCharTok{+}
     \FunctionTok{gg}\NormalTok{(}\AttributeTok{data=}\NormalTok{myBoundary,}\AttributeTok{color=}\StringTok{\textquotesingle{}turquoise\textquotesingle{}}\NormalTok{)}\SpecialCharTok{+}  
  \FunctionTok{gg}\NormalTok{(myMesh4)}\SpecialCharTok{+}
  \FunctionTok{gg}\NormalTok{(myPoints) }\SpecialCharTok{+}
  \FunctionTok{coord\_fixed}\NormalTok{() }\SpecialCharTok{+}
  \FunctionTok{ggtitle}\NormalTok{(}\StringTok{"Points"}\NormalTok{)}
\end{Highlighting}
\end{Shaded}

\includegraphics{2D-Template-analysis_files/figure-latex/unnamed-chunk-18-1.pdf}

\hypertarget{iii.2.-the-model}{%
\subsubsection{III.2. The model}\label{iii.2.-the-model}}

First, specify spatial correlation structure. The following is an
example using a Matern correlation structure with a PC prior.

\begin{Shaded}
\begin{Highlighting}[]
\NormalTok{myCorrelation}\OtherTok{\textless{}{-}}\FunctionTok{inla.spde2.pcmatern}\NormalTok{(myMesh, }\AttributeTok{prior.range =} \FunctionTok{c}\NormalTok{(}\DecValTok{5}\NormalTok{, }\FloatTok{0.01}\NormalTok{), }\AttributeTok{prior.sigma =} \FunctionTok{c}\NormalTok{(}\FloatTok{0.1}\NormalTok{, }\FloatTok{0.01}\NormalTok{))}
\end{Highlighting}
\end{Shaded}

Then, define the model. The model formula requires the explicit name
`coordinates' to recognise the mesh information that it will receive
later, but can use the user-defined `mySmooth()' to specify the spatial
error term.

\begin{Shaded}
\begin{Highlighting}[]
\NormalTok{mySpdeComp}\OtherTok{\textless{}{-}}\NormalTok{coordinates}\SpecialCharTok{\textasciitilde{}}\FunctionTok{mySmooth}\NormalTok{(coordinates, }\AttributeTok{model=}\NormalTok{myCorrelation) }\SpecialCharTok{+} \FunctionTok{Intercept}\NormalTok{(}\DecValTok{1}\NormalTok{)}
\end{Highlighting}
\end{Shaded}

The \texttt{lcgp} models is fitted as follows:

\begin{Shaded}
\begin{Highlighting}[]
\NormalTok{mySpdeFit}\OtherTok{\textless{}{-}}\FunctionTok{lgcp}\NormalTok{(mySpdeComp, }\AttributeTok{data=}\NormalTok{myPoints, }\AttributeTok{samplers=}\NormalTok{myBoundary, }\AttributeTok{domain=}\FunctionTok{list}\NormalTok{(}\AttributeTok{coordinates=}\NormalTok{myMesh))}
\end{Highlighting}
\end{Shaded}

Summary statistics:

\begin{Shaded}
\begin{Highlighting}[]
\FunctionTok{summary}\NormalTok{(mySpdeFit)}
\end{Highlighting}
\end{Shaded}

\begin{verbatim}
## inlabru version: 2.5.2
## INLA version: 22.05.07
## Components:
##   mySmooth: Model types main='spde', group='exchangeable', replicate='iid'
##   Intercept: Model types main='linear', group='exchangeable', replicate='iid'
## Likelihoods:
##   Family: 'cp'
##     Data class: 'SpatialPointsDataFrame'
##     Predictor: coordinates ~ .
## Time used:
##     Pre = 0.813, Running = 8.91, Post = 0.6, Total = 10.3 
## Fixed effects:
##            mean  sd 0.025quant 0.5quant 0.975quant mode kld
## Intercept 1.111 0.6     -0.099    1.121       2.27   NA   0
## 
## Random effects:
##   Name     Model
##     mySmooth SPDE2 model
## 
## Model hyperparameters:
##                    mean    sd 0.025quant 0.5quant 0.975quant mode
## Range for mySmooth 2.12 0.242      1.688     2.10       2.64   NA
## Stdev for mySmooth 1.10 0.095      0.931     1.10       1.30   NA
## 
## Deviance Information Criterion (DIC) ...............: 509.41
## Deviance Information Criterion (DIC, saturated) ....: -16267.64
## Effective number of parameters .....................: -837.43
## 
## Watanabe-Akaike information criterion (WAIC) ...: 1584.57
## Effective number of parameters .................: 138.94
## 
## Marginal log-Likelihood:  -1259.87 
##  is computed 
## Posterior summaries for the linear predictor and the fitted values are computed
## (Posterior marginals needs also 'control.compute=list(return.marginals.predictor=TRUE)')
\end{verbatim}

\hypertarget{iii.3-inference}{%
\subsubsection{III.3 Inference}\label{iii.3-inference}}

Plotting fixed effect parameters

\begin{Shaded}
\begin{Highlighting}[]
\FunctionTok{plot}\NormalTok{(mySpdeFit, }\StringTok{"Intercept"}\NormalTok{)}
\end{Highlighting}
\end{Shaded}

\includegraphics{2D-Template-analysis_files/figure-latex/unnamed-chunk-23-1.pdf}

Plotting spatial random effects Plots of the individual parameters

\begin{Shaded}
\begin{Highlighting}[]
\NormalTok{spde.range }\OtherTok{\textless{}{-}} \FunctionTok{spde.posterior}\NormalTok{(mySpdeFit, }\StringTok{"mySmooth"}\NormalTok{, }\AttributeTok{what =} \StringTok{"range"}\NormalTok{)}
\NormalTok{spde.logvar }\OtherTok{\textless{}{-}} \FunctionTok{spde.posterior}\NormalTok{(mySpdeFit, }\StringTok{"mySmooth"}\NormalTok{, }\AttributeTok{what =} \StringTok{"log.variance"}\NormalTok{)}
\NormalTok{range.plot }\OtherTok{\textless{}{-}} \FunctionTok{plot}\NormalTok{(spde.range)}
\NormalTok{var.plot }\OtherTok{\textless{}{-}} \FunctionTok{plot}\NormalTok{(spde.logvar)}
\FunctionTok{multiplot}\NormalTok{(range.plot, var.plot)}
\end{Highlighting}
\end{Shaded}

\includegraphics{2D-Template-analysis_files/figure-latex/unnamed-chunk-24-1.pdf}

Plots of the correlation and covariance functions

\begin{Shaded}
\begin{Highlighting}[]
\NormalTok{corplot }\OtherTok{\textless{}{-}} \FunctionTok{plot}\NormalTok{(}\FunctionTok{spde.posterior}\NormalTok{(mySpdeFit, }\StringTok{"mySmooth"}\NormalTok{, }\AttributeTok{what =} \StringTok{"matern.correlation"}\NormalTok{))}
\NormalTok{covplot }\OtherTok{\textless{}{-}} \FunctionTok{plot}\NormalTok{(}\FunctionTok{spde.posterior}\NormalTok{(mySpdeFit, }\StringTok{"mySmooth"}\NormalTok{, }\AttributeTok{what =} \StringTok{"matern.covariance"}\NormalTok{))}
\FunctionTok{multiplot}\NormalTok{(covplot, corplot)}
\end{Highlighting}
\end{Shaded}

\includegraphics{2D-Template-analysis_files/figure-latex/unnamed-chunk-25-1.pdf}

\hypertarget{iii.4-model-predictions}{%
\subsubsection{III.4 Model predictions}\label{iii.4-model-predictions}}

First need to generate the prediction data frame. The `pixels()' command
generates a regular grid of points which can be used for the prediction.
This is stored as a spatial data frame in the user-defined
`myPredFrame'.

\begin{Shaded}
\begin{Highlighting}[]
\NormalTok{myPredFrame}\OtherTok{\textless{}{-}}\FunctionTok{pixels}\NormalTok{(myMesh, }\AttributeTok{nx =} \DecValTok{50}\NormalTok{, }\AttributeTok{ny =} \DecValTok{50}\NormalTok{, }\AttributeTok{mask =} \ConstantTok{FALSE}\NormalTok{)}
\end{Highlighting}
\end{Shaded}

To constrain the predictions to a particular region (e.g.~the boundary
of the mesh), set the mask option in the `pixels()' command to
`mask=myBoundary'.

Now we can generate the predictions.

\begin{Shaded}
\begin{Highlighting}[]
\NormalTok{myPreds}\OtherTok{\textless{}{-}}\FunctionTok{predict}\NormalTok{(mySpdeFit, myPredFrame,}\SpecialCharTok{\textasciitilde{}} \FunctionTok{exp}\NormalTok{(mySmooth }\SpecialCharTok{+}\NormalTok{ Intercept))}
\end{Highlighting}
\end{Shaded}

Note that multiple functions and linear predictors can be predicted
simultaneously, under different names.

\begin{Shaded}
\begin{Highlighting}[]
\NormalTok{myPreds}\OtherTok{\textless{}{-}}\FunctionTok{predict}\NormalTok{(mySpdeFit, myPredFrame,  }
                 \SpecialCharTok{\textasciitilde{}} \FunctionTok{data.frame}\NormalTok{(}\AttributeTok{myLambda =} \FunctionTok{exp}\NormalTok{(mySmooth }\SpecialCharTok{+}\NormalTok{ Intercept),}
                              \AttributeTok{myLoglambda =}\NormalTok{ mySmooth }\SpecialCharTok{+}\NormalTok{ Intercept)}
\NormalTok{                )}
\end{Highlighting}
\end{Shaded}

We can visualize multiple aspects of the predictions

Plotting intensity and log-intensity surfaces

\begin{Shaded}
\begin{Highlighting}[]
\NormalTok{pl1 }\OtherTok{\textless{}{-}} \FunctionTok{ggplot}\NormalTok{() }\SpecialCharTok{+}
  \FunctionTok{gg}\NormalTok{(myPreds}\SpecialCharTok{$}\NormalTok{myLambda) }\SpecialCharTok{+}
  \FunctionTok{gg}\NormalTok{(myBoundary) }\SpecialCharTok{+}
  \FunctionTok{ggtitle}\NormalTok{(}\StringTok{"LGCP fit to Points"}\NormalTok{, }\AttributeTok{subtitle =} \StringTok{"(Response Scale)"}\NormalTok{) }\SpecialCharTok{+}
  \FunctionTok{coord\_fixed}\NormalTok{()}
\NormalTok{pl2 }\OtherTok{\textless{}{-}} \FunctionTok{ggplot}\NormalTok{() }\SpecialCharTok{+}
  \FunctionTok{gg}\NormalTok{(myPreds}\SpecialCharTok{$}\NormalTok{myLoglambda) }\SpecialCharTok{+}
  \FunctionTok{gg}\NormalTok{(myBoundary) }\SpecialCharTok{+}
  \FunctionTok{ggtitle}\NormalTok{(}\StringTok{"LGCP fit to Points"}\NormalTok{, }\AttributeTok{subtitle =} \StringTok{"(Linear Predictor Scale)"}\NormalTok{) }\SpecialCharTok{+}
  \FunctionTok{coord\_fixed}\NormalTok{()}
\FunctionTok{multiplot}\NormalTok{(pl1, pl2, }\AttributeTok{cols =} \DecValTok{2}\NormalTok{)}
\end{Highlighting}
\end{Shaded}

\includegraphics{2D-Template-analysis_files/figure-latex/unnamed-chunk-29-1.pdf}

Alternatively, plotting maps of median, lower 95\% and upper 95\%
density surfaces as follows (assuming that the predicted intensity is in
object \texttt{myLambda}).

\begin{Shaded}
\begin{Highlighting}[]
\FunctionTok{ggplot}\NormalTok{() }\SpecialCharTok{+}
  \FunctionTok{gg}\NormalTok{(}\FunctionTok{cbind}\NormalTok{(myPreds}\SpecialCharTok{$}\NormalTok{myLambda, }\FunctionTok{data.frame}\NormalTok{(}\AttributeTok{property =} \StringTok{"q0.500"}\NormalTok{)), }\FunctionTok{aes}\NormalTok{(}\AttributeTok{fill =}\NormalTok{ median)) }\SpecialCharTok{+}
  \FunctionTok{gg}\NormalTok{(}\FunctionTok{cbind}\NormalTok{(myPreds}\SpecialCharTok{$}\NormalTok{myLambda, }\FunctionTok{data.frame}\NormalTok{(}\AttributeTok{property =} \StringTok{"q0.025"}\NormalTok{)), }\FunctionTok{aes}\NormalTok{(}\AttributeTok{fill =}\NormalTok{ q0}\FloatTok{.025}\NormalTok{)) }\SpecialCharTok{+}
  \FunctionTok{gg}\NormalTok{(}\FunctionTok{cbind}\NormalTok{(myPreds}\SpecialCharTok{$}\NormalTok{myLambda, }\FunctionTok{data.frame}\NormalTok{(}\AttributeTok{property =} \StringTok{"q0.975"}\NormalTok{)), }\FunctionTok{aes}\NormalTok{(}\AttributeTok{fill =}\NormalTok{ q0}\FloatTok{.975}\NormalTok{)) }\SpecialCharTok{+}
  \FunctionTok{coord\_equal}\NormalTok{() }\SpecialCharTok{+}
  \FunctionTok{facet\_wrap}\NormalTok{(}\SpecialCharTok{\textasciitilde{}}\NormalTok{property)}
\end{Highlighting}
\end{Shaded}

\includegraphics{2D-Template-analysis_files/figure-latex/unnamed-chunk-30-1.pdf}

\hypertarget{iii.5-estimating-abundance}{%
\subsubsection{III.5 Estimating
abundance}\label{iii.5-estimating-abundance}}

Estimating abundance uses the \texttt{predict} function. As a first step
we need an estimate for the integrated lambda (denoted `Lambda' with an
upper case L). The integration \texttt{weight} values (the quadrature
points) are contained in the \texttt{ipoints} output.

\begin{Shaded}
\begin{Highlighting}[]
\NormalTok{Lambda }\OtherTok{\textless{}{-}} \FunctionTok{predict}\NormalTok{(}
\NormalTok{  mySpdeFit,}
  \FunctionTok{ipoints}\NormalTok{(myBoundary, myMesh),}
  \SpecialCharTok{\textasciitilde{}} \FunctionTok{sum}\NormalTok{(weight }\SpecialCharTok{*} \FunctionTok{exp}\NormalTok{(mySmooth }\SpecialCharTok{+}\NormalTok{ Intercept))}
\NormalTok{)}
\NormalTok{Lambda}
\end{Highlighting}
\end{Shaded}

\begin{verbatim}
##       mean       sd   q0.025   median   q0.975     smin     smax         cv
## 1 667.8037 25.25539 624.4276 668.8877 713.6399 612.9579 724.0196 0.03781859
##        var
## 1 637.8349
\end{verbatim}

Use the median and 95\%iles of this to determine interval boundaries for
estimating the posterior abundance distribution (prediction, not
credible interval).

\begin{Shaded}
\begin{Highlighting}[]
\NormalTok{abundance }\OtherTok{\textless{}{-}} \FunctionTok{predict}\NormalTok{(}
\NormalTok{  mySpdeFit, }\FunctionTok{ipoints}\NormalTok{(myBoundary, myMesh),}
  \SpecialCharTok{\textasciitilde{}} \FunctionTok{data.frame}\NormalTok{(}
    \AttributeTok{N =} \DecValTok{500}\SpecialCharTok{:}\DecValTok{800}\NormalTok{,}
    \FunctionTok{dpois}\NormalTok{(}\DecValTok{500}\SpecialCharTok{:}\DecValTok{800}\NormalTok{,}
      \AttributeTok{lambda =} \FunctionTok{sum}\NormalTok{(weight }\SpecialCharTok{*} \FunctionTok{exp}\NormalTok{(mySmooth }\SpecialCharTok{+}\NormalTok{ Intercept))}
\NormalTok{    )}
\NormalTok{  )}
\NormalTok{)}
\end{Highlighting}
\end{Shaded}

Can get the quantiles of the posterior for abundance via

\begin{Shaded}
\begin{Highlighting}[]
\FunctionTok{inla.qmarginal}\NormalTok{(}\FunctionTok{c}\NormalTok{(}\FloatTok{0.025}\NormalTok{, }\FloatTok{0.5}\NormalTok{, }\FloatTok{0.975}\NormalTok{), }\AttributeTok{marginal =} \FunctionTok{list}\NormalTok{(}\AttributeTok{x =}\NormalTok{ abundance}\SpecialCharTok{$}\NormalTok{N, }\AttributeTok{y =}\NormalTok{ abundance}\SpecialCharTok{$}\NormalTok{mean))}
\end{Highlighting}
\end{Shaded}

\begin{verbatim}
## [1] 596.0300 669.6573 743.4536
\end{verbatim}

\ldots{} the mean via

\begin{Shaded}
\begin{Highlighting}[]
\FunctionTok{inla.emarginal}\NormalTok{(identity, }\AttributeTok{marginal =} \FunctionTok{list}\NormalTok{(}\AttributeTok{x =}\NormalTok{ abundance}\SpecialCharTok{$}\NormalTok{N, }\AttributeTok{y =}\NormalTok{ abundance}\SpecialCharTok{$}\NormalTok{mean))}
\end{Highlighting}
\end{Shaded}

\begin{verbatim}
## [1] 669.8262
\end{verbatim}

and plot posteriors:

\begin{Shaded}
\begin{Highlighting}[]
\NormalTok{abundance}\SpecialCharTok{$}\NormalTok{plugin\_estimate }\OtherTok{\textless{}{-}} \FunctionTok{dpois}\NormalTok{(abundance}\SpecialCharTok{$}\NormalTok{N, }\AttributeTok{lambda =}\NormalTok{ Lambda}\SpecialCharTok{$}\NormalTok{mean)}
\FunctionTok{ggplot}\NormalTok{(}\AttributeTok{data =}\NormalTok{ abundance) }\SpecialCharTok{+}
  \FunctionTok{geom\_line}\NormalTok{(}\FunctionTok{aes}\NormalTok{(}\AttributeTok{x =}\NormalTok{ N, }\AttributeTok{y =}\NormalTok{ mean, }\AttributeTok{colour =} \StringTok{"Posterior"}\NormalTok{)) }\SpecialCharTok{+}
  \FunctionTok{geom\_line}\NormalTok{(}\FunctionTok{aes}\NormalTok{(}\AttributeTok{x =}\NormalTok{ N, }\AttributeTok{y =}\NormalTok{ plugin\_estimate, }\AttributeTok{colour =} \StringTok{"Plugin"}\NormalTok{))}
\end{Highlighting}
\end{Shaded}

\includegraphics{2D-Template-analysis_files/figure-latex/unnamed-chunk-35-1.pdf}

\hypertarget{iv.-factor-glm-with-spde-spatial-term}{%
\subsection{IV. Factor GLM with SPDE spatial
term}\label{iv.-factor-glm-with-spde-spatial-term}}

This will borrow the same (Matern) correlation structure as defined in
part III above \texttt{myCorrelation}. The combined model is defined as
follows. Note the removal of the intercept under the factor model
`factor\_full'.

\begin{Shaded}
\begin{Highlighting}[]
\NormalTok{mySpdeGlmComp }\OtherTok{\textless{}{-}}\NormalTok{ coordinates }\SpecialCharTok{\textasciitilde{}}
\SpecialCharTok{{-}}\DecValTok{1} \SpecialCharTok{+}\FunctionTok{vegetation}\NormalTok{(myCovs}\SpecialCharTok{$}\NormalTok{vegetation, }\AttributeTok{model =} \StringTok{"factor\_full"}\NormalTok{) }\SpecialCharTok{+}
  \FunctionTok{mySmooth}\NormalTok{(coordinates, }\AttributeTok{model =}\NormalTok{ myCorrelation)}
\end{Highlighting}
\end{Shaded}

Model is fitted:

\begin{Shaded}
\begin{Highlighting}[]
\NormalTok{fitSpdeGlm }\OtherTok{\textless{}{-}} \FunctionTok{lgcp}\NormalTok{(mySpdeGlmComp, myPoints, }\AttributeTok{samplers =}\NormalTok{ myBoundary, }\AttributeTok{domain =} \FunctionTok{list}\NormalTok{(}\AttributeTok{coordinates =}\NormalTok{ myMesh))}
\end{Highlighting}
\end{Shaded}

Spatial plot of the fitted (median) intensity surface:

\begin{Shaded}
\begin{Highlighting}[]
\NormalTok{df }\OtherTok{\textless{}{-}} \FunctionTok{pixels}\NormalTok{(myMesh, }\AttributeTok{mask =}\NormalTok{ myBoundary)}
\NormalTok{int2 }\OtherTok{\textless{}{-}} \FunctionTok{predict}\NormalTok{(fitSpdeGlm , df, }\SpecialCharTok{\textasciitilde{}} \FunctionTok{exp}\NormalTok{(mySmooth }\SpecialCharTok{+}\NormalTok{ vegetation), }\AttributeTok{n.samples =} \DecValTok{1000}\NormalTok{)}
\FunctionTok{ggplot}\NormalTok{() }\SpecialCharTok{+}
  \FunctionTok{gg}\NormalTok{(int2, }\FunctionTok{aes}\NormalTok{(}\AttributeTok{fill =}\NormalTok{ median)) }\SpecialCharTok{+}
  \FunctionTok{gg}\NormalTok{(myBoundary, }\AttributeTok{alpha =} \DecValTok{0}\NormalTok{, }\AttributeTok{lwd =} \DecValTok{2}\NormalTok{) }\SpecialCharTok{+}
  \FunctionTok{gg}\NormalTok{(myPoints) }\SpecialCharTok{+}
  \FunctionTok{coord\_equal}\NormalTok{()}
\end{Highlighting}
\end{Shaded}

\includegraphics{2D-Template-analysis_files/figure-latex/unnamed-chunk-38-1.pdf}
\ldots{} and the expected integrated intensity (mean of abundance)

\begin{Shaded}
\begin{Highlighting}[]
\NormalTok{Lambda2 }\OtherTok{\textless{}{-}} \FunctionTok{predict}\NormalTok{(}
\NormalTok{  fitSpdeGlm,}
  \FunctionTok{ipoints}\NormalTok{(myBoundary, myMesh),}
  \SpecialCharTok{\textasciitilde{}} \FunctionTok{sum}\NormalTok{(weight }\SpecialCharTok{*} \FunctionTok{exp}\NormalTok{(mySmooth }\SpecialCharTok{+}\NormalTok{ vegetation))}
\NormalTok{)}
\NormalTok{Lambda2}
\end{Highlighting}
\end{Shaded}

\begin{verbatim}
##       mean       sd   q0.025   median   q0.975     smin     smax        cv
## 1 671.8974 23.10534 627.3458 670.3393 718.0804 623.0767 730.8035 0.0343882
##        var
## 1 533.8569
\end{verbatim}

To look at the contributions to the linear predictor from the SPDE and
from vegetation, we can first generate predictions from those combined,
and individually.

\begin{Shaded}
\begin{Highlighting}[]
\NormalTok{lp2 }\OtherTok{\textless{}{-}} \FunctionTok{predict}\NormalTok{(fitSpdeGlm, df, }\SpecialCharTok{\textasciitilde{}} \FunctionTok{list}\NormalTok{(}
  \AttributeTok{smooth\_veg =}\NormalTok{ mySmooth }\SpecialCharTok{+}\NormalTok{ vegetation,}
  \AttributeTok{smooth =}\NormalTok{ mySmooth,}
  \AttributeTok{veg =}\NormalTok{ vegetation}
\NormalTok{))}
\end{Highlighting}
\end{Shaded}

The function \texttt{scale\_fill\_gradientn} sets the scale for the plot
legend. Here, we set it to span the range of the three linear predictor
components being plotted (medians are plotted by default).

\begin{Shaded}
\begin{Highlighting}[]
\NormalTok{lprange }\OtherTok{\textless{}{-}} \FunctionTok{range}\NormalTok{(lp2}\SpecialCharTok{$}\NormalTok{smooth\_veg}\SpecialCharTok{$}\NormalTok{median, lp2}\SpecialCharTok{$}\NormalTok{smooth}\SpecialCharTok{$}\NormalTok{median, lp2}\SpecialCharTok{$}\NormalTok{veg}\SpecialCharTok{$}\NormalTok{median)}
\NormalTok{csc }\OtherTok{\textless{}{-}} \FunctionTok{scale\_fill\_gradientn}\NormalTok{(}\AttributeTok{colours =} \FunctionTok{brewer.pal}\NormalTok{(}\DecValTok{9}\NormalTok{, }\StringTok{"YlOrRd"}\NormalTok{), }\AttributeTok{limits =}\NormalTok{ lprange)}
\NormalTok{plot.lp2 }\OtherTok{\textless{}{-}} \FunctionTok{ggplot}\NormalTok{() }\SpecialCharTok{+}
  \FunctionTok{gg}\NormalTok{(lp2}\SpecialCharTok{$}\NormalTok{smooth\_veg) }\SpecialCharTok{+}
\NormalTok{  csc }\SpecialCharTok{+}
  \FunctionTok{theme}\NormalTok{(}\AttributeTok{legend.position =} \StringTok{"bottom"}\NormalTok{) }\SpecialCharTok{+}
  \FunctionTok{gg}\NormalTok{(myBoundary, }\AttributeTok{alpha =} \DecValTok{0}\NormalTok{) }\SpecialCharTok{+}
  \FunctionTok{ggtitle}\NormalTok{(}\StringTok{"mySmooth + vegetation"}\NormalTok{) }\SpecialCharTok{+}
  \FunctionTok{coord\_equal}\NormalTok{()}
\NormalTok{plot.lp2.spde }\OtherTok{\textless{}{-}} \FunctionTok{ggplot}\NormalTok{() }\SpecialCharTok{+}
  \FunctionTok{gg}\NormalTok{(lp2}\SpecialCharTok{$}\NormalTok{smooth) }\SpecialCharTok{+}
\NormalTok{  csc }\SpecialCharTok{+}
  \FunctionTok{theme}\NormalTok{(}\AttributeTok{legend.position =} \StringTok{"bottom"}\NormalTok{) }\SpecialCharTok{+}
  \FunctionTok{gg}\NormalTok{(myBoundary, }\AttributeTok{alpha =} \DecValTok{0}\NormalTok{) }\SpecialCharTok{+}
  \FunctionTok{ggtitle}\NormalTok{(}\StringTok{"mySmooth"}\NormalTok{) }\SpecialCharTok{+}
  \FunctionTok{coord\_equal}\NormalTok{()}
\NormalTok{plot.lp2.veg }\OtherTok{\textless{}{-}} \FunctionTok{ggplot}\NormalTok{() }\SpecialCharTok{+}
  \FunctionTok{gg}\NormalTok{(lp2}\SpecialCharTok{$}\NormalTok{veg) }\SpecialCharTok{+}
\NormalTok{  csc }\SpecialCharTok{+}
  \FunctionTok{theme}\NormalTok{(}\AttributeTok{legend.position =} \StringTok{"bottom"}\NormalTok{) }\SpecialCharTok{+}
  \FunctionTok{gg}\NormalTok{(myBoundary, }\AttributeTok{alpha =} \DecValTok{0}\NormalTok{) }\SpecialCharTok{+}
  \FunctionTok{ggtitle}\NormalTok{(}\StringTok{"vegetation"}\NormalTok{) }\SpecialCharTok{+}
  \FunctionTok{coord\_equal}\NormalTok{()}
\FunctionTok{multiplot}\NormalTok{(plot.lp2, plot.lp2.spde, plot.lp2.veg, }\AttributeTok{cols =} \DecValTok{3}\NormalTok{)}
\end{Highlighting}
\end{Shaded}

\includegraphics{2D-Template-analysis_files/figure-latex/unnamed-chunk-41-1.pdf}

The uncertainty in this model can be explored via the coefficient of
variation plot

\begin{Shaded}
\begin{Highlighting}[]
\FunctionTok{ggplot}\NormalTok{() }\SpecialCharTok{+}
  \FunctionTok{gg}\NormalTok{(int2, }\FunctionTok{aes}\NormalTok{(}\AttributeTok{fill =}\NormalTok{ sd }\SpecialCharTok{/}\NormalTok{ mean)) }\SpecialCharTok{+}
  \FunctionTok{gg}\NormalTok{(myBoundary, }\AttributeTok{alpha =} \DecValTok{0}\NormalTok{) }\SpecialCharTok{+}
  \FunctionTok{gg}\NormalTok{(myPoints) }\SpecialCharTok{+}
  \FunctionTok{coord\_fixed}\NormalTok{()}
\end{Highlighting}
\end{Shaded}

\includegraphics{2D-Template-analysis_files/figure-latex/unnamed-chunk-42-1.pdf}

\hypertarget{model-comparisons}{%
\subsubsection{Model comparisons}\label{model-comparisons}}

NOTE: the behaviour of DIC and WAIC is currently a bit unclear, in
particular for experimental mode, and is being investigated./1

\begin{Shaded}
\begin{Highlighting}[]
\NormalTok{knitr}\SpecialCharTok{::}\FunctionTok{kable}\NormalTok{(}\FunctionTok{cbind}\NormalTok{(}
  \FunctionTok{deltaIC}\NormalTok{(myFactorGLM , mySpdeFit, fitSpdeGlm, }\AttributeTok{criterion =} \FunctionTok{c}\NormalTok{(}\StringTok{"DIC"}\NormalTok{)),}
  \FunctionTok{deltaIC}\NormalTok{(myFactorGLM , mySpdeFit, fitSpdeGlm, }\AttributeTok{criterion =} \FunctionTok{c}\NormalTok{(}\StringTok{"WAIC"}\NormalTok{))))}
\end{Highlighting}
\end{Shaded}

\begin{longtable}[]{@{}lrrlrr@{}}
\toprule
Model & DIC & Delta.DIC & Model & WAIC & Delta.WAIC \\
\midrule
\endhead
myFactorGLM & -563.3583 & 0.000 & myFactorGLM & 1373.241 & 0.000 \\
mySpdeFit & 509.4095 & 1072.768 & mySpdeFit & 1584.569 & 211.328 \\
fitSpdeGlm & 597.6010 & 1160.959 & fitSpdeGlm & 1636.821 & 263.580 \\
\bottomrule
\end{longtable}

\end{document}
